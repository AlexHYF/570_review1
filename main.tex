\documentclass{beamer}
\usepackage{algorithm}
\usepackage{algpseudocode}
\usetheme{focus}
\title{Review Session\\ Stable Matching}
\author{Yifei Huang \\ yifeih@usc.edu}

\begin{document}
\begin{frame}
	\maketitle
\end{frame}
\begin{frame}{Crucial Concepts}
	When talking about matching, we usually talk about 2 sets \textbf{A} and \textbf{B} of the same size $n$
	\begin{itemize}
		\item<1-> \textbf{Matching}: A matching $S$ is a set of pairs $(a,b)$ where $a \in A$ and $b \in B$, and no two
			pairs share the same $a$ or $b$.
		\item<2-> \textbf{Perfect Matching}: A matching $S$ is perfect if it is of size $n$. In other word, every $a \in A$
			and $b \in B$ is matched with something.
		\item<3-> \textbf{Unstable Matching}: A matching $S$ is unstable if there exist 2 pairs $(a_1,b_1)$ and $(a_2,b_2)$
			such that $a_1$ prefers $b_2$ than $b_1$ and $b_2$ prefers $a_1$ than $a_2$.
		\item<4-> \textbf{Stable Matching Problem}: Given the preference list of $A$ and $B$, find a perfect matching $S$
			that is not unstable.
	\end{itemize}
\end{frame}

\begin{frame}{G-S Algorithm}
	\begin{block}<1->{Termination}
		G-S algorithm terminates in $O(n^2)$ iterations as each man can only propose at most $n$ times.
	\end{block}
	\begin{alertblock}<2->{Uniqueness}
		G-S algorithm returns a unique solution. But the problem instance might have multiple solutions.
	\end{alertblock}
\end{frame}

\begin{frame}{Question 1}
	Find an instance of stable matching problem where there are multiple solutions and point out the solution that G-S
	algorithm will return.
\end{frame}

\begin{frame}{Answer 1}
	\begin{columns}
	\column{0.6\textwidth}
	\begin{table}
		\centering
		\begin{tabular}{r|cc}
				 &  1st & 2nd \\\hline
			M 1 & W1 & W2 \\
			M 2 & W2 & W1 \\
			W 1 & M2 & M1 \\
			W 2 & M1 & M2 \\
		\end{tabular}
		\caption{Table caption.}
		\label{tab:demo}
	\end{table}
	\column{0.4\textwidth}
	\begin{enumerate}
		\item<1-> \alert{(M1,W1), (M2,W2)}
		\item<2-> (M1,W2), (M2,W1)
	\end{enumerate}
	\end{columns}
\end{frame}

\begin{frame}{Question 2}
	Find an instance of statble matching problem of size $n$, such that G-S algorithm terminates in $O(n)$ iteration.
\end{frame}

\begin{frame}{Answer 2}
	Simple assign each man with different most preferred woman. E.g. $m_i$ prefers $w_i$ the most. In this case G-S
	algorithm will run exactly $n$ iterations as each man will propose to different woman.
\end{frame}
\begin{frame}{Question 3}
	If every man has identical preference list, how many iteration does it take for G-S algorithm to terminate, give the
	precise answer in $n$.
\end{frame}
\begin{frame}{Answer 3}
	With out lose of generality, let's assume that every man's preference list is exactly $(w_1, w_2, ..., w_n)$. G-S
	algorithm returns a stable matching $S = \{(m'_1,w_1), (m'_2,w_2), ..., (m'_n,w_n)\}$. Since every man has the same
	preference list. $m'_i$ must have proposed exactly $i$ times. Then the total number of iteration is $\sum_{i=1}^{n}
	i= \frac{(n+1)n}{2}$.
\end{frame}
\begin{frame}{Question 4}
	Is it true that for every $n > 2$, there exists an instance of stable matching problem that has only one solution?
\end{frame}
\begin{frame}{Answer 4}
	Yes, simple make $m_i$'s $i$-th preferred woman $w_i$ and vice versa. The solution can only be $S = \{(m_i,w_i) |
	\forall i \in [1,n]\}. $Proof is just the extended version of HW1 Q4.
\end{frame}
\end{document}
